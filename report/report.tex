\documentclass[format=sigconf, review=false, nonacm=true]{acmart}

%%
%% \BibTeX command to typeset BibTeX logo in the docs
\AtBeginDocument{%
  \providecommand\BibTeX{{%
    Bib\TeX}}}

\acmDOI{none-existing}

%% These commands are for a PROCEEDINGS abstract or paper.
\acmConference[May]{}{2024}{Ithaca, NY}

\acmISBN{Actually not published}
\setcopyright{none}
%%
%% end of the preamble, start of the body of the document source.
\begin{document}

\title{Normal Map Interpolation GUI for 2D Images}

\author{Barry Lyu}
\affiliation{%
  \institution{fl327}}

\author{Yumeng Jin}
\affiliation{%
  \institution{yj225}
}

\author{Haoxuan (George) Zou}
\affiliation{%
  \institution{hz252}
}

%%
%% By default, the full list of authors will be used in the page
%% headers. Often, this list is too long, and will overlap
%% other information printed in the page headers. This command allows
%% the author to define a more concise list
%% of authors' names for this purpose.
\renewcommand{\shortauthors}{Barry Lyu et al.}

%%
%% The abstract is a short summary of the work to be presented in the
%% article.
\begin{abstract}
  A concise summary of the report, including the purpose, methods, results, and conclusions (about 200-300 words).
\end{abstract}

%%
%% Keywords. The author(s) should pick words that accurately describe
%% the work being presented. Separate the keywords with commas.
\keywords{CS 6682, Lab Cat}
  \settopmatter{printacmref=false, printccs=false, printfolios=false}
%%
%% This command processes the author and affiliation and title
%% information and builds the first part of the formatted document.
\maketitle

\section{Introduction}
Background: Explain the problem your project addresses.\\
Motivation: Why is this problem significant? Discuss the challenges artists face when manually creating normal maps.\\
Objective: Clearly state the goal of your project.\\
Overview: Briefly describe the structure of the report.

\section{Implementation}
Technical Details: Discuss the technologies, libraries, and frameworks used.\\
Code Structure: Provide an outline of the main modules and functions in your code.\\
Algorithms: Explain any key algorithms or processes in detail, particularly those for normal interpolation and rendering.

\section{Methods}
System Architecture: Describe the overall design of your app, including major components and their interactions.\\
Segmentation and Masking: Explain how users can segment images and create masks within your app.\\
Normal Map Creation: Detail the process of adding normals by clicking and dragging arrows.\\
Light Source Definition: Describe how users can define light sources.\\
Rendering Engine: Explain the rendering process based on the defined lights and normals.\\
User Interface: Provide an overview of the user interface design and its usability features.

\section{Results}
Evaluation Metrics: Define the metrics used to evaluate your app (e.g., usability, performance).\\
User Testing: Present the results of any user testing, including feedback from artists.\\
Performance Analysis: Provide data on the app’s performance, such as rendering times and accuracy of the normal maps.

\section{Discussion}
Interpretation of Results: Discuss what your results mean in the context of your objectives.\\
Strengths and Weaknesses: Identify the strengths and potential limitations of your app.\\
Comparison with Related Work: Compare your results with those of existing tools and studies mentioned in the related work section.


%%
%% The acknowledgments section is defined using the "acks" environment
%% (and NOT an unnumbered section). This ensures the proper
%% identification of the section in the article metadata, and the
%% consistent spelling of the heading.
\begin{acks}
To Professor Abe Davis, for teaching us everything that enabled us to complete this project.
\end{acks}

%%
%% The next two lines define the bibliography style to be used, and
%% the bibliography file.
\bibliographystyle{ACM-Reference-Format}
\bibliography{sample-base}


%%
%% If your work has an appendix, this is the place to put it.
\appendix

\section{Related Work}

\subsection{What?}

Literature Review: Summarize existing research and tools related to normal map creation and real-time rendering for 2D images.\\
Comparison: Highlight how your project differs from or improves upon these existing solutions.

\end{document}
\endinput 